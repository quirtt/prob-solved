\documentclass[11pt]{scrartcl}

\usepackage{cancel}
\usepackage[utf8]{inputenc}
\usepackage{mathtools}
\usepackage[sexy]{evan}

\title{12SLC2}
\author{Himadri Mandal}

\begin{document}
\maketitle

\section{Solution}
\begin{soln}
	I claim the answer is $\floor{\frac{2n-1}{5}}$.
	\raggedright

	\textbf{Maximality}:
		Call $f(n)$ the maximum number of possible disjoin pairs.
		Double counting on the sum of elements of pairs and bounding gives
		$$\frac{n\cdot(n+1)}2 - \frac{(n-f(n))(n-f(n)+1)}2 - \frac{(4f(n)(f(n)+1)}2 -f(n)) \geq 0$$
		$$\implies f(n) \leq \floor{\frac{2n-1}5}$$
	\textbf{Construction}:
		For $n = 5q + 3$, consider the following pairs
		$$( 4q+2,1),(4q+3,3) \cdots (3q+2,2q+1) \text{ and } (3q+1,2)\cdots (2q+2,2q)$$.

		This clearly also works for $5m+4,5m+5, 5m+6,5m+7$. So we are done.
\end{soln}
\end{document}