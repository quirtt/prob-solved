\documentclass[11pt]{scrartcl}

\usepackage{cancel}
\usepackage[utf8]{inputenc}
\usepackage{mathtools}
\usepackage[sexy]{evan}

\title{Z6597084}
\author{Himadri Mandal}

\begin{document}
\maketitle

\section{Solution}
\begin{soln}
  \raggedright
  \[ Q(x) \stackrel{\text{def}}= P(x)P(x+1) - P(x^2 + x + 1) \] 
  $Q$ is a polynomial and is $= 0$ for all reals, this means $Q(x) \equiv 0$. 
  So, $P(x)P(x+1) = P(x^2 + x + 1) \ \forall \ x \in \CC$

  Let $\alpha$ be a root of $P$, then clearly $\alpha^2 + \alpha + 1, \alpha^2 - \alpha + 1$ are also roots.
  \begin{claim*}
    Only possible root is $\pm i$. 
  \end{claim*}
  \begin{proof}
    Assume FTSOC, there exists a root which is not $= \pm i$
    We know $|x^2 -x + 1| = |-x^2 +x -1|$, so using triangle inequality, 
    \[ |x^2 + x + 1| + |-x^2 + x -1| \geq 2|x| \]

    Unless $|x^2 + x + 1| = |x^2 - x + 1| = |x|$, we run into problems. It is easy to see that this happens
    only for $x = \pm i$. So, the only possible roots are $\pm i$.
  \end{proof}
  \hrule

  \bigskip
  Let $P = (x-i)^{a}(x+i)^{b}$ which always works. So, $\boxed{P = \{\{(x-i)^a(x+i)^b : a,b \geq 0\}, 0, 1\}}$ are the only solutions.
  
\end{soln}

\end{document}
