\documentclass[11pt]{scrartcl}

\usepackage{cancel}
\usepackage[utf8]{inputenc}
\usepackage{mathtools}
\usepackage[sexy]{evan}

\title{20BXMO1}
\author{Himadri Mandal}

\begin{document}
\maketitle

\section{Solution}
\begin{soln}
  \raggedright
  This is equivalent to finding $d$ such that $\exists P \ \in \ZZ[X]$ with 
  \[S(P) = \#_{\ZZ}(P-1) + \#_{\ZZ}(P+1) \geq  d+1 \]
  where $\#_{\ZZ}(P)$ denotes the number of distinct integral roots of $P$. 
  \begin{claim*}
    $S(P) \leq 4$ which also means $d \leq 3$.   
  \end{claim*}
  \begin{proof}
    \[ P(x) + 1 =  Q(x) (x-\alpha_1)^{a_1}(x-\alpha_2)^{a_2}\cdots (x-\alpha_k)^{a_k} \]
    \[ P(x) - 1 =  Q(x) (x-\alpha_1)^{a_1}(x-\alpha_2)^{a_2}\cdots (x-\alpha_k)^{a_k} - 2\]
    If $P(x_0)-1 = 0$ with $x_0 \in \ZZ$, 
    \[ Q(x_0)(x_0 - \alpha_1)^{a_1}(x_0 - \alpha_2)^{a_2} \cdots (x_0 - \alpha_k)^{a_k} = 2 \]
    So, either $Q(x_0) = \pm 2, k \leq 2$ or $Q(x_0) = \pm 1, k \leq 3$, translating we get,

    \textbf{Case 1:}
    \[ \pm2 x^{a_1}(x+c)^{a_2} \]
    $S(P) \leq 3$   

    \textbf{Case 2:}
    \[ \pm 1 x^{a_1}(x+c_1)^{a_2}(x+c_2)^{a_3}\]
    Sorting $x,x+c_1,x+c_2$ and translating accordingly such that $x > x+c_1 > x+c_2$, but then $x = 2 \text{ or } 1$ which 
    determines the root itself. Thus, $S(P) \leq 4$ 
  \end{proof}  \hrule

  \bigskip

  \textbf{Construction:} 
  \begin{itemize}
    \item $d = 1: x$
    \item $d = 2: 2x^2 -1$
    \item $d = 3: -x(x-1)(x-3)-1$
  \end{itemize} 

\end{soln}

\end{document}
