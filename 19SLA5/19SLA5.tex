\documentclass[11pt]{scrartcl}

\usepackage{cancel}
\usepackage[utf8]{inputenc}
\usepackage{mathtools}
\usepackage[sexy]{evan}

\title{19SLA5}
\author{Himadri Mandal}

\begin{document}
\maketitle

\section{Solution}
\begin{soln}
  We will prove the version where $x_i$ are complex numbers. Let the expression be $P/V$.
  The denominator is the \textbf{Vandermonde Determinant}
    \[ V = \prod_{i > j} (x_i - x_j) \]
  Call $Q_i = V \cdot \prod_{j \neq i} \frac{1 - x_ix_j}{x_i - x_j}$
\begin{claim*}
    $P(x_1,x_2,\cdots,x_n)$ is divisible by the denominator $V$. 
  \end{claim*}
  \begin{proof}
        So we would be done if we show $P = 0$ if $x_i = x_j$ for $i \neq j$.
    
    \textbf{Assume} $x_1 = x_k$, 
    clearly $Q_2 = \cdots = Q_{k-1} = Q_{k+1} \cdots = Q_n = 0$,
    \[ Q_1 = (-1)^{n-1}(1-x_1x_2)(1-x_1x_3)\cdots(1 - x_1x_n)\cdot \frac{V}{\prod_{j > 1} (x_j - x_1)}\]
    \[ Q_k = (-1)^{n-k}\frac{\prod_{j} (1 - x_kx_j)}{1-x_j^2}\cdot \frac{V}{\prod_{j > k} (x_j - x_k) \prod_{j < k}(x_k - x_j)} \]
    (There are no divide by 0 issues because we first divide and then equate the variables.) 
    We get, $Q_k = (-1)^{k-1+k-2} \cdot Q_1$, which proves the claim. Other cases follow similarly.
  \end{proof}
  \hrule

  \bigskip
  \begin{claim*}
    $P(x_1,x_2,\cdots,x_n) = C \cdot \prod_{i > j} (x_i - x_j)$, where $C \in \RR$ and is constant for
    all variable $x_i$'s.
  \end{claim*}
  \begin{proof}
    Assume $P$ is nonzero. We already know 
    \[ P = V \cdot R \]
    where $V$ is the denominator and $R \in \RR[X_1,X_2,\cdots,X_n]$. It is well known that $\deg_{x_i}(V) = n-1$.
    Thus, $\deg_{x_i}(V \cdot R) \geq n-1$ while $\deg_{x_i}(P) \leq n-1$, because we can choose the terms
    in lexicographic order of degrees. Equality holds only when
    $\deg_{x_i}(P) = n-1$ and $\deg_{x_i}(R) = 0$
  \end{proof}   
  \hrule

  \bigskip
  \newpage
  To finish we have to show this holds true for 1 set of values, take $x_{n+1} = 1$,
  \[ \sum_{i,n+1} \prod_{i \neq j} \frac{1 - x_ix_j}{x_i - x_j} = 
  (-1)^n\sum_{i,n}\prod_{i \neq j} \frac{1 - x_ix_j}{x_i-x_j} + 1 =
  (-1)^n \cdot \frac{1 - (-1)^n}{2} + 1 = \frac{1+(-1)^n}{2}
  \]
  As the base case is trivial, we are done by induction. 
\end{soln}
\end{document}
