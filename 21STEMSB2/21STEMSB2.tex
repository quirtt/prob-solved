\documentclass[11pt]{scrartcl}

\usepackage{cancel}
\usepackage[utf8]{inputenc}
\usepackage{mathtools}
\usepackage[sexy]{evan}

\title{21STEMSB2}
\author{Himadri Mandal}

\begin{document}
\maketitle

\section{Solution}
\begin{soln}
  Let $\{p_1,\cdots,p_k\}$ be the set of first $k$ odd primes with $p_k < 10^{100} < p_{k+1}$. If $P(0) \neq 0$,
  choose a big $n_0$, $P(2^{n_0}) = 2^{\nu_2(P(0))}\prod_ip_i^{\alpha_i}$
  Note that $p_1^{\alpha_1} \mid\mid P(2^{n_0})$, now $p_1^{\alpha_1} \mid\mid P(2^{n_0 + \varphi(p_1^{\alpha_1+1})})$
  because \[ a_k(2^{n_0 + \varphi(p_1^{\alpha_1+1})})^k \equiv a_k(2^{n_0})^k \cdot (2^{\varphi(p_1^{\alpha_1+1})})^k \equiv a_k (2^{n_0})^k \pmod{p_1^{\alpha_1+1}} \]
  and $\varphi(p_1^{\alpha_1}) | \varphi(p_1^{\alpha_1 + 1})$. Therefore, 
  \[ P(2^{n_0}) = P(2^{n_0 + \lambda\varphi(\prod_ip_i^{\alpha_i + 1})}). \]
  This is not possible so $P(0) = 0$. Now, $Q(x) \overset{\text{def}}= \frac{P(x)}{x}$, if $Q(0) \neq 0$ we could derive the same contradiction,
  continuing this way we see that $P(x) = x^n$ is the only solution.
\end{soln}
\end{document}
