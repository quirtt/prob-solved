\documentclass[11pt]{scrartcl}

\usepackage{cancel}
\usepackage[utf8]{inputenc}
\usepackage{mathtools}
\usepackage[sexy]{evan}

\title{16SLC5}
\author{Himadri Mandal}

\begin{document}
\maketitle

\section{Solution}

Sorry for the wording Evan, even though I know you wouldn't mind, couldn't resist myself $>$\_$<$.
\begin{soln}
	
I claim the answer is $n-3$ when $n$ is odd and $n-2$ otherwise. 
$n-3$ is quite easy to prove, just notice that there can be no
intersecting diagonals, so the trivial triangulation is the best
we can do. The construction for evens is to triangulate in order from a 
vertex and then join the farthest 2 vertices. 


\bigskip

Now lets prove the statement for evens.
Call any diagonal which intersects in the polygon 
``\text{\color{red}hot}". If we have 2 intersecting
\text{\color{red}hot} diagonals $m_1, m_2$, then clearly every other \text{\color{red}hot} diagonal
is parallel to either $m_1$ or $m_2$.

\begin{claim}
	The longest {\color{red}hot} diagonal in an orientation cannot be a part of a \text{\color{red}hot} rectangle.
\end{claim}
\begin{proof}
	If it were then, the arc \textit{not} in the region of the biggest \text{\color{red}hot} rectangle 
	would have lesser no. of diagonals because we can always make a new \text{\color{red}hot} diagonal which
	doesn't disturb the triangulation of the arc. 
\end{proof}
Assume there are $\gamma$ \text{\color{red}hot} diagonals.
Now, notice that there are atmost $n - (\gamma+2)$ \text{\color{blue}virgin} vertices which don't touch a \text{\color{red}hot} diagonal.
\begin{claim}
	Each \text{\color{blue}virgin} vertex contribute atmost 1 diagonal.
\end{claim}
\begin{proof}
	If there are $\geq 2$ \text{\color{blue}virgin} vertices in that \text{\color{blue}
	virgin} arc then clearly a triangulation does the job.
	There cannot be a single \text{\color{blue}virgin} vertex in any such arc because
	if there was one it would mean the adjacent vertices are \text{\color{red}
	hot-touching} but that would contradict $\color{green}\textbf{Claim 1.1}$.
\end{proof}

So we have atmost $\gamma + (n - (\gamma+2)) = n-2$ diagonals.
\end{soln}

\end{document}