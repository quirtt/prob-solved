\documentclass[11pt]{scrartcl}

\usepackage{cancel}
\usepackage[utf8]{inputenc}
\usepackage{mathtools}
\usepackage[sexy]{evan}

\title{H1883820}
\author{Himadri Mandal}

\begin{document}
\maketitle

\section{Solution}
\begin{soln}
  \raggedright
  Let $P(x) = x^3 + a_2 x^2 + a_1 x + a_0$ be a polynomial which has roots $a,b,c$. 
  $p_i = a^i + b^i + c^i$, so we have
  \[ p_3 = -a_{2}^3 + 3a_{2}a_{1} - 3a_{0} \]
  \[ p_5 = -a_{2}^5 + 5a_{2}^3 a_{1} - 5a_{2}^2 a_{0} - 5a_{1}^2a_{2} + 5a_{1}a_{0} \]
  If $p_3 = p_1^3 = -a_2^3 \implies a_2 a_1 = a_0$, thus, 
  \[ p_5 = -a_2^5 + 5(a_2^3a_1 -a_2^2a_0) - 5(a_1^2a_2 - a_1a_0) = -a_2^5 \]
  On the other hand, if $p_5 = p_1^5 \implies a_1a_2(a_2^2 - a_1) = a_0(a_2^2 - a_1)$
  
  Thus, $a_1a_2 = a_0$ or $a_2^2 = a_1$, first case is trivial. For the second case, 
  we have $(a+b+c)^2 = ab + bc + ca \implies a = -b, b = -c, c = -a \implies a = b = c = 0$, for
  which the statement obviously holds true.
\end{soln}

\end{document}
