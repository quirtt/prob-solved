\documentclass[11pt]{scrartcl}

\usepackage{cancel}
\usepackage[utf8]{inputenc}
\usepackage{mathtools}
\usepackage[sexy]{evan}

\title{18BAMO4}
\author{Himadri Mandal}

\begin{document}
\maketitle

\section{Solution}
\raggedright
\begin{soln}
	
Take a prime $p \mid abc$ and let $x = \nu_p(a)$,s
$y = \nu_p(b)$, $z = \nu_p(c)$.
It is enough to prove $x+y+z \equiv 0 \pmod{3}$.

Notice that if $x=y=z$ we are done, so assume this is not true. Then $\nu_p(a/b) = x-y$, $\nu_p(b/c) = y-z$, $\nu_p(c/a) = z-x$. One of them have to be negative, which is not possible. But we have $\nu_p(a/b+b/c+c/a) \ge 0$, so we conclude that the two smallest numbers among $\{x-y, y-z, z-x\}$ must be equal.

To finish WLOG assume that if $x-y = y-z$, then $2y = x+z$ and so $x+y+z \equiv 0 \pmod 3$.

\end{soln}
\end{document}