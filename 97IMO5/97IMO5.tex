\documentclass[11pt]{scrartcl}

\usepackage{cancel}
\usepackage[utf8]{inputenc}
\usepackage{mathtools}
\usepackage[sexy]{evan}

\title{97IMO5}
\author{Himadri Mandal}

\begin{document}
\maketitle

\section{Solution}
\newcommand\rad
{\operatorname{rad}}
\begin{soln}
  Firstly notice that $\rad(a) = \rad(b)$, this is quite easy to prove.
  \begin{claim*}
    $b^3 | a$
  \end{claim*}
  \begin{proof}
    Assume $a < b$, this leads to a clear contradiction. So $a \geq b$. For 
    $a = b$ we easily get $a = b =1$. Now assume $a > b$, 
    \[ a^{b^2} = b^a < a^a \implies a > b^2\]
    \[ a^{b^2} = b^a = (b^2)^{\frac{a}{2}} < a^{\frac{a}2} \implies a > 2b^2 \]
    \[ \frac{\nu_p(a)}{\nu_p(b)} = \frac{a}{b^2} > 2 \]
    Thus, $b^2 | a$, but using this in the previous equation again we would have 
    \[ \frac{\nu_p(a)}{\nu_p(b)} = \frac{a}{b^2} \geq 3 \]
    So, $b^3 | a$.
  \end{proof}
  \hrule

  \bigskip
  Set $a = kb^3$. The original equation reduces to $k = b^{bk - 3}$. To finish,
  \[ k = b^{bk - 3} = b^{b^{bk - 2} - 3} = b^{b^{b^{bk - 2} - 2} -3} \]
  If $bk - 2 \geq 3$ then the size of the RHS increases unboundedly. Thus, $bk = \{1,2,3,4\}$
  Checking these we get $(a,b) \in (1,1),(16,2),(27,3)$ are the only solutions.
\end{soln} 
\end{document}
