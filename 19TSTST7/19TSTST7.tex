\documentclass[11pt]{scrartcl}

\usepackage{cancel}
\usepackage[utf8]{inputenc}
\usepackage{mathtools}
\usepackage[sexy]{evan}

\title{19TSTST7}
\author{Himadri Mandal}

\begin{document}
\maketitle

\section{Solution}
\begin{soln}
  \raggedright
  Call $\lambda = \lcm(1,2,\dots,10^{100})$. Let $S = \{1,2,\dots,\lambda\}.$ 
  \begin{claim*}
    $f(x)$ is periodic. 
  \end{claim*}
  \begin{proof}
    Set $y = x + \lambda$, we get 
    \[ \gcd(f(x),f(y)) = \gcd(f(x),\lambda) = f(x) \implies f(x) | f(x + \lambda) \forall x \in \ZZ, \]
    but now
    setting $x = y + \lambda$, we get
    \[ \gcd(f(y + \lambda),f(y)) = \gcd(f(y+\lambda),\lambda) = f(y+\lambda) \implies f(y+\lambda)|f(y) \forall y \in \ZZ\]
    Therefore, $f(x) = f(x + \lambda) \forall x \in \ZZ$, which means it is periodic.
  \end{proof}
  
  \hrule

  \bigskip
  Now to finish, fix a prime $p$, choose $x=x_p$ such that $\nu_p(f(x_p)) = \displaystyle\max_{x \in S}\nu_p(f(x))$
  Therefore, 
  \[ \nu_p(f(y)) = \nu_p(\gcd(f(x_p),f(y))) = \nu_p(\gcd(f(x_p),x_p-y))\]
  
  Notice that $y$'s of the form 
  \[ y \equiv x_p \pmod{p^{\nu_p(f(x_p))}} \]
  clearly have maximal $\nu_p$, so performing CRT on 
  \[ y \equiv x_p \pmod{p^{\nu_p(f(x_p))}} \]
  over all primes $p|\lcm(f(1),\dots,f(\lambda))$. Let $\alpha$ be a number that satisfies all those congruences.

  Note that $f(x) = \gcd(f(\alpha),x-\alpha)$. Now, taking $m = -\alpha,n=f(\alpha)$. We are done.
\end{soln}
\end{document}
