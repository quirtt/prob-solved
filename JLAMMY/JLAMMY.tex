\documentclass[11pt]{scrartcl}

\usepackage{cancel}
\usepackage[utf8]{inputenc}
\usepackage{mathtools}
\usepackage[sexy]{/home/quirtt/Documents/OTIS/evan}

\title{JLAMMY}
\author{Himadri Mandal}

\begin{document}
\maketitle

\section{Solutions}
\begin{soln}
\raggedright
  I claim that 2026 is the actual lower limit.

  \textbf{Construction}:
  Consider the set $$A = \{-44,-43,...,45\}$$

  It is easy to see that sum of elements of nonempty subsets of $A$ achieves all numbers $x : -990 \leq x \leq 1035$.

  \textbf{Minimality}:

  Note that there are atleast $\frac{m(m+1)}{2}$ different sums in $\{a_1,...,a_m\}$ with $0 < a_1 < a_2 < ... < a_m$ or the other way around. It is easy to see that if $0 \in A$ then we would have lower num of sums. So if there are $a$ positives and $b$ negatives in $A$ then $S \geq \frac{a(a+1)}{2} + \frac{b(b+1)}{2}$ and $a+b = 89$. Which achieves minimum at $2026$.
\end{soln}
\end{document}