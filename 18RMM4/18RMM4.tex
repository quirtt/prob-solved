\documentclass[11pt]{scrartcl}

\usepackage{cancel}
\usepackage[utf8]{inputenc}
\usepackage{mathtools}
\usepackage[sexy]{evan}

\title{18RMM4}
\author{Himadri Mandal}

\begin{document}
\maketitle

\section{Solution}
\begin{prob}
 Let $a,b,c,d$ be positive integers such that $ad \neq bc$ and $gcd(a,b,c,d)=1$. Let $S$ be the set of values attained by $\gcd(an+b,cn+d)$ as $n$ runs through the positive integers. Show that $S$ is the set of all positive divisors of some positive integer. 
\end{prob}
\begin{soln}
  \raggedright
  Denote $F(a,b,c,d,n) = \gcd(an+b,cn+d)$. Therefore $F(a,b,c,d,n) = F(c,d,a,b,n)$.
  Note that \[ \gcd(an+b, cn+d) = \gcd((a-c)n + (b-d),cn+d) \] 
  which means $F(a,b,c,d,n) = F((a-c),(b-d),c,d,n)$.
  WLOG assume $a+b \geq c+d$
  \[ F(a,b,c,d,n) = F((a-c),(b-d),c,d,n) \text{ Here, } [(a-c)+(b-d)] < (a+b) \]
  The sum $(a+b)$ is strictly decreasing, perform this operation until $(a+b) < (c+d)$
  in which case flip the order $(a,b,c,d,n) \to (c,d,a,b,n)$ such that the inequality $(c+d) \geq (a+b)$
  is maintained. Also, $\gcd(a,b,c,d) = \gcd(\gcd(a,c),\gcd(b,d)) = \gcd(\gcd(a-c,c),\gcd(b-d,d)) = \gcd(a-c,b-d,c,d) = 1$. 

  Now note that the sum $(a+b+c+d)$ is also decreasing. Therefore, $c=0$ at some point.
  So it suffices to show that
  \begin{claim*}
    $ad \neq 0$ and $\gcd(a,b,d)=1$. Let $S$ be the set of values attained by 
    $\gcd(an+b,d)$ as $n$ runs through the positive integers. $S$
    is the set of all positive divisors of some positive integer.     
  \end{claim*}
  \begin{proof}
    Let $\mathbb{G}=\{ \gcd(an+b,d): n\geq 1\}$.
    WLOG assume $\gcd(a,b)=1$. Let $g = \gcd(a,d)$ $a=g\alpha,d=g\beta$, we would be done if we show that $q | \beta \implies q \in \mathbb{G}$
    \[ g\alpha n+b \equiv q \pmod{\beta} \]
    \[ \text{Take } n \equiv \dfrac{q-b}{g\alpha} \pmod{\beta} \] 
    which proves the claim.
  \end{proof}
  \hrule

  \bigskip
  And we are done.
\end{soln}

\end{document}
