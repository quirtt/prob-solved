\documentclass[11pt]{scrartcl}

\usepackage{cancel}
\usepackage[utf8]{inputenc}
\usepackage{mathtools}
\usepackage[sexy]{evan}

\title{PROMYS5}
\author{Himadri Mandal}

\begin{document}
\maketitle

\section{Thoughts, Ideas, Claims and Proofs}
Denote $f(m,n)$ to be the answer for $m \text{x} n$ board.

Confession: I am using geogebra to make grids

Thought: $f(m, n) = \gcd(m,n) \cdot f(m/\gcd(m,n), n/\gcd(m,n))$

Proof: Umm lets see. if it is a $g \cdot x \text{ x } g \cdot y$ board then 
by similarity we can find $g$ $x \text{ x } y$ grids on the diagonal line with 
the line intersecting both the corners. Thus, $f(gx, gy) = g \cdot f(x,y)$. 

So we limit to thinking about $f(x,y)$ with $\gcd(x,y) = 1$. Now I am going to think
about the $3 \text{ x } 5$ grid. Ok so, thinking about "when" there is a change in the 
current "box" of the line by thinking about the $y$ coords
as the $x$ coords change, 
$$(0, \frac35, 1, \frac65, \frac95, 2,\frac{12}5, 3)$$

\end{document}
