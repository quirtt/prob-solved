\documentclass[11pt]{scrartcl}

\usepackage{cancel}
\usepackage[utf8]{inputenc}
\usepackage{mathtools}
\usepackage[sexy]{evan}

\title{06IMO5}
\author{Himadri Mandal}

\begin{document}
\maketitle

\section{Solution}
\begin{soln}
Huh. Very nice.
\begin{lemma*}
  For any $P \in \ZZ[X]$, and $m \in \ZZ$ with 
  $P^k(m) = m \implies k \in \{1,2\}$ 
\end{lemma*}
\begin{proof}
  \[ m - P(m) | P(m) - P(P(m)) | \cdots | P^{k-1}(m) - m | m - P(m)\]
  So, $\dfrac{P^{i+1}(m) - P^i(m)}{P^i(m) - P^{i-1}(m)} = \{-1,1\} \forall i$, if this
  equals $-1$ atleast once then, 
  \[P^{c}(m) = P^{c-2}(m) \implies P(P(m)) = P^{gk+2}(m) = P^{gk}(m) = m\]
  So $P^{k}(m) - P^{k-1}(m) = P^{k-1}(m) - P^{k-2}(m) = \cdots = P(m) - m$, which means
  $m = P^k(m) = m+k(P(m) - m)$, so we are done.
\end{proof}
  Using this lemma we reduce the problem to $k = 2$. 
  Define $\#_{\ZZ}(P(x))$ to be the number of distinct
  integer roots of $P(x)$. Let $S_1 = \{a_i\}_{i \leq k_1}$ be the set of integers with
  cyclicity 1 and $S_2 = \{b_i\}_{i \leq k_2}$ be the set of integers 
  with cyclicity 2. If $S_2$ is empty we are done, so assume not.

\begin{claim*}
  $\#_{\ZZ}(P(P(x)) - x) \leq n$
\end{claim*}
\begin{proof}
  If $|S_1| \neq 0$ then
  \[ b_i - a_1 | P(b_i) - a_1 | b_i - a_1 \]
  $\implies P(b_i) - a_1 = a_1 - b_i \implies P(b_i) = 2a_1 - b_i = 2a_j - b_i$
  thus $S_1 = \{a_1\}$. But then clearly $P(x)-2a_1+x$ has all of 
  $S_1 \cup S_2$. 
  
  If $|S_1| = 0$ then
  \[ b_1 < \cdots < b_{k_2} \]
  \[ b_i - b_1 | P(b_i) - P(b_1) | b_i - b_1 \]
  Assume there exist $p > q > 1$ then 
  $P(b_p) - P(b_1) = b_p - b_1$ and $P(b_q) - P(b_1) = b_1 - b_q$
  $P(b_p) - P(b_q) = |b_p + b_q - 2b_1| = b_p - b_q$ which is absurd.
  Therefore, $\{b_i\}$ are all either the roots of
  $P(x)+x - (P(b_1)+b_1)$ or $P(x)-x - (P(b_1)-b_1)$.
  So we are done by fundamental theorem.
\setqed{$\blacksquare$}\end{proof} \setqed{$\square$}
\hrule

\bigskip
\setqed{} \end{soln} \setqed{$\square$}
\end{document}
