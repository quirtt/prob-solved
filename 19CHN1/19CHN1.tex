\documentclass[11pt]{scrartcl}

\usepackage{cancel}
\usepackage[utf8]{inputenc}
\usepackage{mathtools}
\usepackage[sexy]{evan}

\title{19CHN1}
\author{Himadri Mandal}

\begin{document}
\maketitle

\section{Solution}
\begin{soln}
	
\raggedright
The answer is $-512 \le S \le 288$.
288 is achieved at $a=b=4$ and $c=d=e=-1$ and 
the -512 is achieved at $a=9$ and $b=c=d=e=-1$.
Lets prove the inequalities now, 

Let $x_1 = a+b$, $x_2 = c+d$, $x_3 = e+a$, $x_4 = b+c$, $x_5 = d+e$.
Then the conditions become
\[ \sum_i x_i = 10 \quad\text{ and }\quad
	x_i + x_{i+1} \le 6 \; \forall i \]
We also have $-2 \le x_i \le 8$ for each $i$,
and $S = x_1 x_2 x_3 x_4 x_5$.
Let $f(i) = \begin{cases}
	0 & x_i \geq 0 \\
	1 & \text{otherwise} \\
\end{cases}
$.
$T = \cycsum f(i)$

Lets do some casework.
\begin{itemize}
	\ii If $T = 0$
	then $S \ge 0$ and $S \le 2^5 = 32$ by AM-GM.

	\ii If $T=1$ $S \le 0$,
	and $|S| \le 2 \cdot 3^2 \cdot 3^2 = 162$
	by AM-GM (since $x_1 x_2 \le 9$, $x_3 x_4 \le 9$).

	\ii If $T=2$, so $S \ge 0$.
	Two of the nonnegative $x_i$'s must be adjacent,
	say $x_1$ and $x_2$, thus $x_1 x_2 \le 9$.
	So $|S| \le 2^2 \cdot 9 \cdot 8 = 288$.

	Equality is achieved at $(3,3,-2,8,-2)$.

	\ii If $T=3$, so $S \le 0$.
	In that case, $|S| \le 2^3 \cdot 8^2 = 512$.

	Equality is achieved at $(-2,-2,-2,8,8)$.

	\ii If $T=4$, so $S \ge 0$.
	Then $|S| \le 2^4 \cdot 8 = 128$.

	\ii $T=5$ is not possible
	since $\sum_i x_i = 10$.
\end{itemize}
Therefore, we deduce that \[ -512 \le S \le 288 \] as desired.
\end{soln}
\end{document}